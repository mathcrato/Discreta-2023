\item
		Seja $f :\mathbb{R}\to \mathbb{R}-\{0\}$  uma função satisfazendo às condições:
		\[
		\begin{cases}
						f(x+y) =f(x)\cdot f(y)\ \forall x,y\in\mathbb{R}\\
						f(x)\neq 1, \forall x\in\mathbb{R}-\{0\}
		\end{cases}
		\]
Podemos afirmar:
		\begin{enumerate}
				\item $f$ pode ser ímpar. %Falsa
				\item $f(0)=1$
				\item $f$ é injetiva.
				\item  $f$ não é sobrejetiva, pois $f(x)>0,\ \forall x\in \mathbb{R}$.
		\end{enumerate}

		\Sol{
				\begin{enumerate}
						\item ({\bf Falso}) Uma função é impar se para todo $x$, $f(-x)=-f(x)$.

								Tomando $x=y=\frac{x}{2}$, obtemos:
								\begin{eqnarray*}
										f(\frac{x}{2}+\frac{x}{2}) =  f(\frac{x}{2})\cdot f(\frac{x}{2})\\
										f(x) =  [f(\frac{x}{2})]^2
								\end{eqnarray*}
Ou seja, para todo $x\in\R$, $f$ é sempre positivo e portanto não pode ser ímpar. 
						\item ({\bf Verdadeiro}) Tome $x=y=0$
								\begin{eqnarray*}
										f(0+0) = f(0)\cdot f(0)\\
										f(0) = f(0)\cdot f(0)
								\end{eqnarray*}
								Como $f(0)\neq 0$ para todo $x$, obtemos $f(0) = \frac{f(0)}{f(0)}=1$
						\item ({\bf Verdadeiro}) Seja 
								\begin{eqnarray*}
										x\neq y \Rightarrow x-y &\neq& 0\\
										f(x-y) &\neq& 1\ \text{pois}\ f(x)\neq 1, \forall x\neq 0\\
										f(x)f(-y) &\neq& 1\\
										f(x)  &\neq& \frac{1}{f(-y)}
								\end{eqnarray*}
		Agora, note que $f(y-y) = f(y)f(-y)\Rightarrow f(0)=f(y)f(-y)\Rightarrow f(y)=\frac{1}{f(-y)}$. 
		Portanto, $f(x) \neq \frac{1}{f(-y)}=f(y)$. Logo, $f$ é injetiva.
						\item ({\bf Verdadeiro}) Pelo item (a), $f$ é sempre positivo, mas o domínio de $f$ pelo enunciado
								é $\R-\{0\}$ portanto $f$ não é sobrejetiva pois não assume valores negativos.
				\end{enumerate}
		}
