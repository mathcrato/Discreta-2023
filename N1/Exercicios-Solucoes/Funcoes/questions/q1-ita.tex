\item Uma função $f$ de variável real satisfaz a
  condição $f(x+1)=f(x)+f(1)$, qualquer que seja o valor
  da variável $x$. Sabendo-se que $f(2)=1$, podemos
  concluir que $f(5)$ é igual a:
 
  \Sol{
Tomando $x=1$:
		\begin{eqnarray*}
				f(1+1) &= & f(1)+f(1)\\
				f(2) &= & 2f(1)\\ 
				1    &= & 2f(1)\ \Rightarrow f(1)=\frac{1}{2}
		\end{eqnarray*}
Tomando $x=2$:
		\begin{eqnarray*}
				f(2+1) &= & f(2)+f(1)\\
				f(3) &= & 1+\frac{1}{2}\ \Rightarrow f(3)=\frac{3}{2}
		\end{eqnarray*}
Tomando $x=3$:
		\begin{eqnarray*}
				f(3+1) &= & f(3)+f(1)\\
				f(4) &= & \frac{3}{2}+\frac{1}{2}\ \Rightarrow f(4)=\frac{4}{2}=2
		\end{eqnarray*}
Tomando $x=4$:
		\begin{eqnarray*}
				f(4+1) &= & f(4)+f(1)\\
				f(5) &= & 2+\frac{1}{2}\ \Rightarrow f(5)=\frac{5}{2}
		\end{eqnarray*}


}
