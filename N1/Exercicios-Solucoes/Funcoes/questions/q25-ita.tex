\item
%(ITA )
 Considere $g: \{a,b,c\} \to \{a,b,c\}$ uma função tal que $g(a)=b$ e $g(b)=a$. Então, temos:
%\begin{multicols}{5}
\begin{enumerate}
\item a equação $g(x)=x$ tem solução se, e somente se, $g$ é injetora. %Resp
\item $g$ é injetora, mas não é sobrejetora.
\item $g$ é sobrejetora, mas não é injetora.
\item se $g$ não é sobrejetora, então $g(g(x)) = x$ para todo $x$ em $\{a,b,c\}$
\item n.d.r.a.
\end{enumerate}
%\end{multicols}

\Sol{
		\begin{enumerate}
				\item Como $g(a) = b$ e $g(b) = a$ e se a função é injetiva, então necessariamente $g(c) = c$ , logo $g(x) = x$ tem solução. 
				\item  Se $g$ é injetiva, $g(c) = c$ e então $Img = CD$, logo também será sobrejetiva.
				\item  Se $g$ é sobrejetiva, devemos ter $g(c) = c$ pois $g(a)=b$ e $g(b)=a$, logo também será injetiva.
				\item  Se $g$ não é sobrejetiva $g(c) = b$ ou $g(c) = a$. 
						Temos que:
					\begin{eqnarray*}
				\text{Se}\ g(a)=b \Rightarrow g(g(a)) = g(b) = a\ \text{Verdadeiro}\\	
				\text{Se}\ g(b)=a \Rightarrow g(g(b)) = g(a) = b\ \text{Verdadeiro}\\
				\text{Se}\  g(c) = b \Rightarrow g(g(c)) = g(b) = a\ \text{Falso}\\
				\text{Se}\  g(c) = a \Rightarrow g(g(c)) = g(a) = b\ \text{Falso}
					\end{eqnarray*}
%letra a
		\end{enumerate}
}
