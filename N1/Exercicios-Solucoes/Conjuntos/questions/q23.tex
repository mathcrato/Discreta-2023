\item %(ITA 2000) 
		Denotemos por $n(X)$ o número de elementos de um conjunto finito $X$. Sejam $A$, $B$ e $C$ conjuntos tais que
$n(A \cup B)= 8$, $n(A \cup C)= 9$, $n(B \cup C)= 10$, $n(A \cup B \cup C) = 11$ e $n (A \cap B \cap C) = 2$.
Então, $n(A) + n(B) + n(C)$ é igual a
\begin{multicols}{5}
\begin{enumerate}
\item $ 11$
\item $ 14$
\item $ 15$
\item $ 18$ %D correta
\item $ 25$
\end{enumerate}
\end{multicols}
