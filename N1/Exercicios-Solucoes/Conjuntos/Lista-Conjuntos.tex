\documentclass[11pt,a4paper]{article}
\usepackage{moesio} 
\usepackage{amssymb}
%-------------------------------------------------------------
\usepackage[url,doi,style=numeric,backend=biblatex]{biblatex}
\addbibresource{ref.bib}
%-------------------------------------------------------------
\newcommand{\nome}{\bf Moésio M. de Sales\footnote{moesio@ifce.edu.br}}
\newcommand{\titu}{Conjuntos}
\newcommand{\subtitu}{Exercícios}
\newcommand{\disc}{Matemática Discreta}
\newcommand{\curso}{Sistemas de Informação}
\newcommand{\inst}{IFCE}
%-------------------------------------------------------------
\usepackage{tikz}
\usetikzlibrary{fit, shapes.geometric, arrows}
%-------------------------------------------------------------
\renewcommand\textbullet{\ensuremath{\bullet}}
%\renewcommand{\qed}{\hfill\blacksquare}
\newcommand{\qedwhite}{\hfill \ensuremath{\Box}}
%-------------------------------------------------------------
\newcommand{\qst}[1]{\noindent
\colorbox{blue!10}{\begin{minipage}[t]{.85\textwidth}
		 #1  \end{minipage}}}
\newcommand{\Sol}[1]{\noindent
\colorbox{red!10}{\begin{minipage}[t]{.88\textwidth}
\begin{sol} #1 \qedwhite\end{sol}\end{minipage}}}
\newcommand{\dest}[1]{\noindent
\colorbox{red!10}{\begin{minipage}[t]{.85\textwidth}
\begin{center} #1 \end{center} \end{minipage}}}
%-------------------------------------------------------------
\newcommand{\Z}{{\mathbb Z}}
\newcommand{\N}{{\mathbb N}}
\newcommand{\R}{{\mathbb R}}
%-------------------------------------------------------------
\begin{document}
{\Large
\begin{center} \titu\\ \disc\\ \curso \\  \nome\end{center}
}
%-------------------------------------------------------------
\hfill   %\today\\[2mm]
\hrule\ 
%-------------------------------------------------------------
%-------------------------------------------------------------
\section{\sc Problemas Conjuntos}

Estude\cite{gersting2004fundamentos,Ed1976teoria,scheinerman2003mat}

\bexer
\item Se um conjunto $A$ tem $1024$ subconjuntos, então o cardinal de $A$, ou número de elementos de $A$, é:
  \begin{multicols}{5}
	\begin{enumerate}
\item $5$
\item $6$
\item $7$
\item $9$
\item $10$
	\end{enumerate}
  \end{multicols}

  \Sol{
		  \begin{eqnarray*}
				  2^n &= & 1024\\
				  2^n &= & 2^{10}\ \Rightarrow 	  n=10
		  \end{eqnarray*}
  }

\item  $35$ estudantes estrangeiros vieram ao Brasil. $16$ visitaram Manaus; $16$, S. Paulo e $11$, Salvador. Desses estudantes, $5$ visitaram Manaus e Salvador e , desses $5$, $3$ visitaram também São Paulo. O número de estudantes que visitaram Manaus ou São Paulo foi:
		\begin{multicols}{5}
\begin{enumerate}
\item $29$
\item $24$
\item $11$
\item $8$
\item $5$
\end{enumerate}
		\end{multicols}

\item  Seja $E = \{\Delta\}$ . Determine $\mathcal{P}(\mathcal{P}(E))$.
\item Determine $\mathcal{P}(\mathcal{P}(\mathcal{P}( \varnothing )))$.
\item Prove que $A \subset B \Leftrightarrow \mathcal{P}(A) \subset \mathcal{P}(B).$

\item
Sejam dois conjuntos, $X$ e $Y$, e a operação $\Delta$, definida por
$X \Delta Y = (X - Y) \cup (Y -X)$.
Justifique todos os ítens.
\begin{enumerate}
\item $(X \Delta Y) \cap (X \cap Y) = \varnothing$
\item $(X \Delta Y) \cap (X - Y) = \varnothing$
\item $(X \Delta Y) \cap (Y - X) = \varnothing$
\item $(X \Delta Y) \cup (X - Y) = X$
\item $(X \Delta Y) \cup (Y - X) = X$
\end{enumerate}

\Sol{
		\begin{enumerate}
		\item \begin{eqnarray*}
		(X \Delta Y) \cap (X \cap Y) &=& [(X-Y)\cup(Y-X)]\cap (X\cap Y)\\
		                             &=& [(X-Y)\cap (X\cap Y)]\cup [(Y-X)\cap (X\cap Y)]\\
		                           &=& \varnothing \cup \varnothing \\
		                           &=& \varnothing
		\end{eqnarray*}
		\end{enumerate}
}

\item Dados os conjuntos $A$ e $B$, seja $X$ um conjunto com as 
seguintes propriedades:
\begin{enumerate}
\item[I.] $X \supset A$ e $X \supset B$,
\item[II.] Se $Y \supset A$ e $Y \supset B$ então $Y \supset X$
\end{enumerate}
Prove que: $X=A\cup B$.

\Sol{
		Provaremos primeiro ($A\cup B \subset X$):

		Seja 
		\begin{eqnarray*}
				x\in A\cup B &\Rightarrow & x\in A\ \text{ou}\ x\in B\\
				\text{Como, por (I),}\ && A\subset X \ \text{e}\  B\subset X\\
				\text{Em qualquer caso:}		     &\Rightarrow & x\in X
		\end{eqnarray*}
		Portanto, $A\cup B\subset X$.

		Provaremos, agora: ($X\subset A\cup B$):
		Note que: 
		\begin{eqnarray*}
				A\cup B \supset A\ \text{e}\ A\cup B \supset B \ \text{Pela propriedade (II)}\Rightarrow 	A\cup B\supset X
		\end{eqnarray*}
		Portanto, $X\subset A\cup B$.

		Assim, dado que $X\subset A\cup B$ e $X\supset A\cup B$, então $X= A\cup B$.

}

\item %elon
Sejam $A,B\subset E$. 
\begin{enumerate}
\item Prove que $A\cap B=\varnothing $ se, somente se, $A\subset B^c$.
\item Prove que $A\cup B=E $ se, somente se, $A^c\subset B$.
\end{enumerate}
\item Dados os intervalos $A=[-1,3)$, $B=[1,4]$, $C=[2,3)$, $D=(1,2]$ e $E=(0,2]$ dizer se $0$ pertence a $((A-B)-(C\cap D))-E$.
\item %elon
Dados $A,B\ \subset E$, prove que $A\subset B$ se, somente se, $A\cap B^c =\varnothing$.
\item 
Se $A,X \subset E$ são tais que $A\cap X=\varnothing$ e $A\cup X=E$, prove que $X=A^c$.
\item 
Seja $A\vartriangle B=(A-B)\cup (B-A)$. Prove que $A\vartriangle B=A\vartriangle C$ implica $B=C$.

\Sol{
		{\bf Primeiro:}

$A\vartriangle B=A\vartriangle C \Rightarrow B \subset C$

Note que se $x\in A\vartriangle B$, significa que $x\in A-B$ ou $x\in B-A$, ou seja, $x$ pertence apenas a $A$ ou apenas a $B$.

Seja $x\in B$. Temos dois casos:

\begin{itemize}
		\item Se $x$ pertence apenas a $B$, então $x\in (A-B)\cup (B-A)$. Como $A\vartriangle B=A\vartriangle C$
				neste caso $x\in A\vartriangle C$, logo $x\in(A-C)\cup (C-A)$ e como $x\not\in A$ temos que $x\in C$.

		\item Se $x$ pertence a $B$ e a $A$, então $x\not\in (A-B)\cup (B-A)$.  Como $A\vartriangle B=A\vartriangle C$
				neste caso $x\not\in A\vartriangle C$, logo $x\not\in(A-C)\cup (C-A)$ e como $x\in A$ temos que $x\in C$.
\end{itemize}
 
{\bf Analogamente:}

$A\vartriangle B=A\vartriangle C \Rightarrow B \supset C$

}

\item 
Uma urna contém três bolas vermelhas, duas azuis e uma amarela. Duas
bolas são selecionadas aleatoriamente sem reposição. Sejam os eventos:
\[A = \{\mbox{pelo menos uma bola é vermelha}\}\ B = \{\mbox{pelo menos uma bola é azul}\}\]
Descreva usando a notação de conjuntos os seguintes eventos:
\begin{enumerate}
\item Ambas as bolas são amarelas.
%O evento em questão é que não ocorre $A$ e não ocorre $B$, ou seja, $A^c\cap B^c$
\item Há uma bola vermelha e uma azul.
%Há uma bola vermelha e uma azul na amostra, se e somente se, os eventos $A$ e $B$ ocorrem, ou seja, $A ∩ B$
\end{enumerate}
\Sol{
\begin{enumerate}
\item 
O evento em questão é que não ocorre $A$ e não ocorre $B$, ou seja, $A^c \cap B^c$
\item 
Há uma bola vermelha e uma azul na amostra, se e somente se, os eventos $A$ e
$B$ ocorrem, ou seja, $A \cap B$
\end{enumerate}
}

\item 
Nas mesmas condições do exercício anterior, descreva os seguintes eventos:
\begin{enumerate}
\item $A \cap B^c$ 
\item $ A\cup B$ 
\item $ A \cup B^c$
\end{enumerate}

%A ∩ B c significa que há pelo menos uma bola vermelha e não há bolas azuis.
%A ∪ B representa que há pelo menos uma bola que não é amarela.
%A ∪ B c Há duas bolas azuis na amostra, ou uma vermelha e uma azul ou uma ver-
%melha e uma amarela. Note que isto é precisamente o evento A, pois o evento B c ,
%implica que o evento A ocorre, pois se nào há bolas azuis na amostra, necessaria-
%mente uma delas será vermelha.

\Sol{
		\begin{enumerate}
		\item $A \cap B^c=\{\mbox{Significa que há pelo menos uma bola vermelha e não há bolas azuis.} \}$ 
		\item $ A\cup B =\{\mbox{Representa que há pelo menos uma bola que não é amarela.}\}$ 
		\item $ A \cup B^c =$
		Há duas bolas azuis na amostra, ou uma vermelha e uma azul ou uma vermelha e uma amarela. 
		Note que isto é precisamente o evento $A$, pois o evento $B^c$, implica que o evento $A$ ocorre, 
		pois se não há bolas azuis na amostra, necessariamente uma delas será vermelha.
		\end{enumerate}
}

\eexerc

%-------------------------------------------------------------
\printbibliography
%-------------------------------------------------------------
\end{document}
