\documentclass[11pt,a4paper]{article}
\usepackage{moesio} %estilo Moésio
%=========================================================================================
\usepackage[url,doi,style=numeric,backend=biber]{biblatex}
\addbibresource{ref.bib}
%=======================================================================================
%\newcommand{\nomer}{\bf Me. Moésio}
\newcommand{\nome}{\bf Moésio M. de Sales\footnote{moesio@ifce.edu.br}}
\newcommand{\nomet}{\bf Moésio M. de Sales}
\newcommand{\titu}{ Princípio de Indução e Função}
\newcommand{\subtitu}{ Recorrências Lineares}
\newcommand{\disc}{Matemática Discreta}
\newcommand{\curso}{Sistemas de Informação}
\newcommand{\inst}{IFCE}
%=================================================================================
\usepackage{tikz}
\usetikzlibrary{fit, shapes.geometric, arrows}
\begin{document}
{\Large
\begin{center} \titu\\ \disc\\ \curso \\  \nome\end{center}
}
%==========================================================================
\hfill   %\today\\[2mm]
\hrule\ 
%==========================================================================


\section{\sc Problemas Conjuntos}


{\exe 
		Mostre as seguintes fórmulas por Indução\cite{gersting2004fundamentos,demidovitch1965problems}
\begin{enumerate}
\item $
1+2+\cdots + n =\frac{n(n+1)}{2}
$
\item $2^n>n$, para todo natural $n$.
\end{enumerate}
}



\bexer
\item Uma função $f$ de variável real satisfaz a
  condição $f(x+1)=f(x)+f(1)$, qualquer que seja o valor
  da variável $x$. Sabendo-se que $f(2)=1$, podemos
  concluir que $f(5)$ é igual a:
 
  \Sol{
Tomando $x=1$:
		\begin{eqnarray*}
				f(1+1) &= & f(1)+f(1)\\
				f(2) &= & 2f(1)\\ 
				1    &= & 2f(1)\ \Rightarrow f(1)=\frac{1}{2}
		\end{eqnarray*}
Tomando $x=2$:
		\begin{eqnarray*}
				f(2+1) &= & f(2)+f(1)\\
				f(3) &= & 1+\frac{1}{2}\ \Rightarrow f(3)=\frac{3}{2}
		\end{eqnarray*}
Tomando $x=3$:
		\begin{eqnarray*}
				f(3+1) &= & f(3)+f(1)\\
				f(4) &= & \frac{3}{2}+\frac{1}{2}\ \Rightarrow f(4)=\frac{4}{2}=2
		\end{eqnarray*}
Tomando $x=4$:
		\begin{eqnarray*}
				f(4+1) &= & f(4)+f(1)\\
				f(5) &= & 2+\frac{1}{2}\ \Rightarrow f(5)=\frac{5}{2}
		\end{eqnarray*}


}

\item  A função $f:\mathbb{R}\rightarrow \mathbb{R}$  tal que
  $f(x+y)=f(x)+f(y)$, para todo $x$ e $y$. Calcule $f(0)+1$.

  \Sol{
		  Se $x=y=0$:
		  \begin{eqnarray*}
				  f(0+0)= f(0)+f(0) \Rightarrow 
				  f(0)= f(0)+f(0) \Rightarrow 
				  f(0)= 0  
		  \end{eqnarray*}

		  Potanto, 
		  \begin{eqnarray*}
				  f(0)+1 =0+1=1
		  \end{eqnarray*}
  }

\item  Se $f$  uma função de $\mathbb{R}$ em $\mathbb{R}$ definida por
  $f(x)=(x-3)/(x^2+3)$, então a expressão $\frac{f(x)-f(1)}{(x-1)}$,
  para $x\leq 1$,  equivalente a
  \begin{multicols}{2}
	\begin{enumerate}
	  \item  $(x + 3)/2(x^2 + 3)$
	  \item $(x - 3)/2(x^2 + 3)$
	  \item $ (x + 1)/2(x^2 + 3)$
	  \item $ (x - 1)/2(x^2 + 3)$
	  \item $-1/x$
	\end{enumerate}
  \end{multicols}


\item
%(ITA )
 Considere $g: \{a,b,c\} \to \{a,b,c\}$ uma função tal que $g(a)=b$ e $g(b)=a$. Então, temos:
%\begin{multicols}{5}
\begin{enumerate}
\item a equação $g(x)=x$ tem solução se, e somente se, $g$ é injetora. %Resp
\item $g$ é injetora, mas não é sobrejetora.
\item $g$ é sobrejetora, mas não é injetora.
\item se $g$ não é sobrejetora, então $g(g(x)) = x$ para todo $x$ em $\{a,b,c\}$
\item n.d.r.a.
\end{enumerate}
%\end{multicols}

\Sol{
		\begin{enumerate}
				\item Como $g(a) = b$ e $g(b) = a$ e se a função é injetiva, então necessariamente $g(c) = c$ , logo $g(x) = x$ tem solução. 
				\item  Se $g$ é injetiva, $g(c) = c$ e então $Img = CD$, logo também será sobrejetiva.
				\item  Se $g$ é sobrejetiva, devemos ter $g(c) = c$ pois $g(a)=b$ e $g(b)=a$, logo também será injetiva.
				\item  Se $g$ não é sobrejetiva $g(c) = b$ ou $g(c) = a$. 
						Temos que:
					\begin{eqnarray*}
				\text{Se}\ g(a)=b \Rightarrow g(g(a)) = g(b) = a\ \text{Verdadeiro}\\	
				\text{Se}\ g(b)=a \Rightarrow g(g(b)) = g(a) = b\ \text{Verdadeiro}\\
				\text{Se}\  g(c) = b \Rightarrow g(g(c)) = g(b) = a\ \text{Falso}\\
				\text{Se}\  g(c) = a \Rightarrow g(g(c)) = g(a) = b\ \text{Falso}
					\end{eqnarray*}
%letra a
		\end{enumerate}
}

\item
		 %(ITA 2009)
		Seja $f :\mathbb{R}\to \mathbb{R}-\{0\}$  uma função satisfazendo às condições:
		\[
		\begin{cases}
						f(x+y) =f(x)\cdot f(y)\ \forall x,y\in\mathbb{R}\\
						f(x)\neq 1, \forall x\in\mathbb{R}-\{0\}
		\end{cases}
		\]
Podemos afirmar:
		\begin{enumerate}
				\item $f$ pode ser ímpar. %Falsa
				\item $f(0)=1$
				\item $f$ é injetiva.
				\item  $f$ não é sobrejetiva, pois $f(x)>0,\ \forall x\in \mathbb{R}$.
		\end{enumerate}

\item
%(IME )
		 Definimos a função $f: \mathbb{N} \to \mathbb{N}$ da seguinte forma
\[
		\begin{cases}
		f(0) = 0 \\
		f(1) = 1 \\
		f(2n) = f(n),\ n\geq 1 \\
		f(2n+1) = n^2,\ n\geq 1 \\
		\end{cases}
\]
  Definimos a função $g: \mathbb{N} \to \mathbb{N}$ da seguinte forma: 
  \[g(n) = f(n)\cdot f(n + 1). \]
%\begin{multicols}{5}
  Podemos afirmar que:
\begin{enumerate}
		\item $g$ é uma função sobrejetora.
		\item $g$ é uma função injetora.
		\item $f$ é uma função sobrejetora.
		\item $f$ é uma função injetora.
		\item $g(2018)$ tem mais do que $4$ divisores positivos. 
\end{enumerate}
%\end{multicols}

\Sol{
\begin{enumerate}
		\item Da igualdade $f(2n) = f(n), n \geq 1$ tem-se
				\[
						f(2^k) = f(2^{2k - 1}) = f(2^{2k - 2}) = \cdots = f(2) = f(1) = 1
		\]
		Assim, todas as potências de $2$ tem imagem igual a $1$.
		\item Da igualdade $f(2n + 1) = n^2, n \geq 1$ tem-se
				$f(3) = f(2 \cdot 1 + 1) = 1$. As imagens de todos os demais números ímpares é um quadrado perfeito diferente de $1$, portanto diferente de $3$.
		\item  A imagem de qualquer outro número par que não é potência de $2$, e igual a imagem de um número ímpar e, portanto, diferente de $3$.
		\item Como $3 \in CD(f)$, $3 \in CD(g)$, $3 \in Im(f)$ e $3 \in Im(g)$, nem $f$, nem $g$ são sobrejetoras.
		\item $f(2) = f(1) = 1$, portanto $f$ não é injetora.
				\begin{eqnarray*}
						g(1) = f(1) \cdot f(2) = 1 \cdot 1 = 1\ \text{e}\\
						g(2) = f(2) \cdot f(3) = 1 \cdot 1 = 1 
				\end{eqnarray*}
$g$ não é injetora, pois $g(1) = g(2)$
		\item 
				\begin{eqnarray*}
				g(2018) &= & f(2018) \cdot f(2019) = f(1009) \cdot f(2019) \\
					   &= & 5042 \cdot 10092
				\end{eqnarray*}
				que possui mais do que $4$ divisores positivos
\end{enumerate}
}

\item %(ITA 2000) 
		Denotemos por $n(X)$ o número de elementos de um conjunto finito $X$. Sejam $A$, $B$ e $C$ conjuntos tais que
$n(A \cup B)= 8$, $n(A \cup C)= 9$, $n(B \cup C)= 10$, $n(A \cup B \cup C) = 11$ e $n (A \cap B \cap C) = 2$.
Então, $n(A) + n(B) + n(C)$ é igual a
\begin{multicols}{5}
\begin{enumerate}
\item $ 11$
\item $ 14$
\item $ 15$
\item $ 18$ %D correta
\item $ 25$
\end{enumerate}
\end{multicols}

\item  Seja $S= \{3,4,5,6,7,8,9,10,11,12\}$. Suponha que seis inteiros sejam escolhidos de $S$. Existem dois inteiros cuja a soma é $15$?
\item  Mostre que para qualquer conjunto de $13$ números escolhidos no intervalo  $[2, 40]$,  existem pelo menos dois inteiros com um divisor comum maior que $1$. 
%{\exe Seja $S= \{3,4,5,6,7,8,9,10,11,12\}$. Suponha que seis inteiros sejam escolhidos de $S$. Existem dois inteiros cuja a soma é $15$?}
%Resposta:
%Sim.  Seja Y o conjunto de todos os pares de inteiros de S que somam 15. Existem cinco elementos em Y , ou seja, Y = {{3, 12}, {4, 11}, {5, 10}, {6, 9}, {7, 8}}, e cada inteiro de S ocorre em exatamente um par.  Seja X o conjunto de seis inteiros escolhidos de S e considere a função de X para Y denida pela regra: P (x) = o par para o qual x pertence. Como X tem seis elementos e Y tem cinco elementos e 6 > 5, então pelo princípio da casa de pombo, P não  uma função injetiva. Assim, P (x1 ) = P (x2 ) para alguns inteiros x1 e x2 em X com x1 6= x2 . Isto signica que x1 e x2 são inteiros distintos no mesmo pair, o que implica x1 + x2 = 15.

%{\exe Mostre que para qualquer conjunto de $13$ números escolhidos no intervalo  $[2, 40]$,  existem pelo menos dois inteiros com um divisor comum maior que $1$. }
%Seja A o conjunto de 13 números escolhidos e seja B o conjunto de todos os números primos no intervalo
%[2, 40], ou seja, B = {2, 3, 5, 7, 11, 13, 17, 19, 23, 29, 31, 37}. Para cada x em A, seja F (x) o menor número
%primo que divide x. Como A tem 13 elementos e B tem 12 elementos, pelo princípio da casa de pombo,
%F não é uma função injetiva. Assim, F (x1) = F (x2 ) para algum x1 6= x2 em A. Pela denição de F , isto
%signica que o menor número primo que divide x1 é igual ao menor número primo que divide x2 . Assim,
%dois números em A, x1 e x2, tem um divisor comum maior que 1.
\input{questions/q5(induc)}
\input{questions/q6(induc)}
\input{questions/q7(induc)}
\input{questions/q8(induc)}
\eexerc
%==========================================================================================
\printbibliography
%==========================================================================================
\end{document}
